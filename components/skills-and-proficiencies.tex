\begin{longtable}{@{}>{\raggedright}p{6.25in} >{\raggedleft}X@{}}

\hangindent=5ex \underline{Scientific programming for data wrangling and visualization:} \texttt{R} (\texttt{base}, \texttt{dplyr}, \texttt{data.table}, \texttt{tidyr}, \texttt{ggplot2}, \texttt{tmap}, \texttt{mapview}) & \tabularnewline
\hangindent=5ex \underline{Geospatial:} Google Earth Engine (JavaScript, Python), \texttt{R} (\texttt{sf}, \texttt{terra}, \texttt{stars}, \texttt{raster},  \texttt{lidR}), Structure from Motion photogrammetry (Pix4Dmapper, Agisoft Metashape), QGIS, CloudCompare & \tabularnewline
\hangindent=5ex \underline{Remote sensing:} Drone operations and data, satellite data (Landsat, MODIS, GOES), multispectral sensors (Micasense RedEdge, DJI Mavic 3 Multispectral), hyperspectral sensors (AVIRIS imaging spectrometer), FAA-licensed Remote Pilot (2017 to present) & \tabularnewline
\hangindent=5ex \underline{Cloud computing:} AWS EC2, Docker, RStudio server & \tabularnewline
\hangindent=5ex \underline{Collaborative coding and version control:} git, GitHub & \tabularnewline
\hangindent=5ex \underline{Inference:} Hierarchical modeling in \texttt{R} using Bayesian frameworks (\texttt{brms}, NIMBLE) and maximum likelihood (\texttt{lme4}), spatial machine learning, interpretable machine learning, population dynamics in \texttt{R} (simulations, integral projection models), geostatistics & \tabularnewline
\hangindent=5ex \underline{Fieldwork:} Vegetation plot establishment, tree stem mapping using laser instruments, GLORIA multi-summit approach & \tabularnewline
\hangindent=5ex \underline{Dynamic documents:} Quarto, RMarkdown, \LaTeX{} & \tabularnewline

\end{longtable}