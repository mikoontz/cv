\begin{longtable}{@{} >{\raggedright}p{5.25in} >{\raggedleft}X @{}}

\hangindent=5ex \textbf{Koontz, Michael J.}, Malcolm P. North, Amy DeCastro, Jennifer K. Balch, and Andrew M. Latimer. Fine-scale drivers of California megafires.  & [\textcolor{blue}{\href{https://github.com/mikoontz/megafire-fine-scale-drivers}{GitHub}}] \tabularnewline

\hangindent=5ex Winsemius, Sara, Yufang Jin, Hugh D. Safford, Michelle C. Agne, Robert A. Andrus, Alina Cansler, Anthony C. Caprio, W. Wallace Covington, Joseph E. Crouse, Calvin Farris, Peter Z. Fule, Brian J. Harvey, Sharon M. Hood, David W. Huffman, Emma J. McClure, John P. Roccaforte, Saba Saberi, Mochael T. Stoddard, Laura Trader, Phillip J. van Mantgem, Micah C. Wright, and \textbf{Michael J. Koontz}. Basal area loss from fire using field-calibrated remote sensing refines western U.S. fire severity measurements. & \tabularnewline

\hangindent=5ex \textbf{Koontz, Michael J.}, Andrea Duane, Katharyn Duffy, and Hugh D. Safford. Fire regime departure reveals vulnerabilities to state change adn opportunities for restoration. & \tabularnewline

\hangindent=5ex DeCastro, Amy, \textbf{Michael J. Koontz}, Malcolm P. North and Andrew M. Latimer. Heterogeneity can promote landscape resilience to extreme fire spread events. & \tabularnewline

\hangindent=5ex DeCastro, Amy, \textbf{Michael J. Koontz}, and Jennifer K. Balch. Local-scale predictors of fire spread across the U.S. & \tabularnewline

\hangindent=5ex \textbf{Koontz, Michael J.}, Zachary L. Steel, Andrew M. Latimer, and Malcolm P. North. Initial wildfire suppression efforts select for more extreme fuel and climate burning conditions in Sierra Nevada forests.  & [\textcolor{blue}{\href{https://github.com/mikoontz/selection-by-suppression}{GitHub}}] \tabularnewline

\hangindent=5ex Provost, Mikaela, Jan Ng, Jessica Rudnick, Linda Estel\'i M\'endez Barrientos, Steven P. Lee, \textbf{Michael J. Koontz}, and Emilio A. Laca. Novel integration of holistic review and statistical analysis to rank applications in an R1 STEM graduate program. & \tabularnewline

\hangindent=5ex Huesca, Margarita, \textbf{Michael J. Koontz}, Alexander Koltunov, Yuhan Huang, Andrew M. Latimer, and Yufang Jin. Tree mortality assessment using imaging spectroscopy data in the Sierra Nevada mountains. & \tabularnewline

% \hangindent=5ex DeCastro, Amy, and \textbf{Michael J. Koontz}. Calculating surface area from a broad-extent, fine-grain digital surface model using Google Earth Engine. & \tabularnewline

\end{longtable}

%\hangindent=5ex Emilio A. Laca, Steven P. Lee, Jan Ng, Mikaela M. Provost, Jessica Rudnick, Linda Mendez-Barriento, Derek J. Young, \textbf{Michael J. Koontz}, Anne Todgham, Ben Sacks, and Elizabeth Sturdy. Holistic review in graduate admissions for an R1 STEM program: I. Statistical method to weight desirable applicant qualities and minimize reviewer effects. & \tabularnewline

% these are some collaborations that will hopefully be revived
%\hangindent=5ex \textbf{Michael J. Koontz} and Jeff C. Schank. An agent based model simulation of the outbreak behavior of the western pine beetle during the 2012 to 2016 Sierra Nevada megadrought. & [\textcolor{blue}{\href{https://github.com/mikoontz/wpb-forest-structure-abm}{GitHub}}] \tabularnewline

%\hangindent=5ex Jens T. Stevens, \textbf{Michael J. Koontz}, and Chhaya M. Werner.
%Local effects of aspect on vegetation productivity in California.  & \tabularnewline
%
%\hangindent=5ex \textbf{Michael J. Koontz} and Ruth A. Hufbauer.
%Several, small introductions of individuals to a novel environment facilitate adaptation by mitigating genetic load. & [\textcolor{blue}{\href{https://github.com/mikoontz/ppp-adaptation}{GitHub}}] \tabularnewline

% This code includes affiliations, but perhaps that is too verbose.
%\textbf{Michael J. Koontz}$^{1}$, Chhaya M. Werner$^{1}$, Stephen E. Fick$^{2}$, Malcolm P. North$^{1,3}$, and Andrew M. Latimer$^{1}$. Local variability of vegetation structure increases forest resilience to wildfire
%$^{1}$University of California, Davis, $^{2}$U.S. Geological Survey, Southwest Biological Science Center, $^{3}$U.S. Forest Service, Pacific Southwest Research Station
%\textcolor{blue}{\href{https://github.com/mikoontz/remote-sensing-resistance}{https://github.com/mikoontz/remote-sensing-resistance}}  & \tabularnewline

%\textbf{Michael J. Koontz}$^{1}$, Andrew M. Latimer$^{1}$, Leif A. Mortenson$^{2}$, Chris J. Fettig$^{2}$, Connie I. Millar $^{2}$, and Malcolm P. North$^{1,2}$ 
%\emph{The effect of spatial variability of forest structure on the severity of a tree-killing insect} 
%$^{1}$University of California, Davis, $^{2}$U.S. Forest Service, Pacific Southwest Research Station 
%\textcolor{blue}{\href{https://github.com/mikoontz/local-structure-wpb-severity}{https://github.com/mikoontz/local-structure-wpb-severity}}  & \tabularnewline

%Jens T. Stevens$^{1}$, \textbf{Michael J. Koontz}$^{2}$, and Chhaya M. Werner$^{2}$ 
%\emph{Local effects of aspect on vegetation productivity in California.}\tabularnewline
%$^{1}$U.S. Geological Survey, Southwest Biological Science Center, $^{2}$University of California, Davis  & \tabularnewline

%\textbf{Michael J. Koontz}$^{1}$ and Ruth A. Hufbauer$^{2}$ 
%\emph{Several, small introductions of individuals to a novel environment facilitate adaptation by mitigating genetic load} 
%$^{1}$University of California, Davis, $^{2}$Colorado State University   & \tabularnewline


